%!TEX root = std.tex
\rSec0[concepts]{Concepts library}

\rSec1[concepts.general]{General}

\pnum
This Clause describes library components that \Cpp{} programs may use to perform
compile-time validation of template arguments and perform function dispatch
based on properties of types. The purpose of these concepts is to establish
a foundation for equational reasoning in programs.

\pnum
The following subclauses describe language-related concepts, comparison
concepts, object concepts, and callable concepts as summarized in
\tref{concepts.summary}.

\begin{libsumtab}{Fundamental concepts library summary}{concepts.summary}
\ref{concepts.equality} & Equality preservation     &                    \\ \hline
\ref{concepts.lang}     & Language-related concepts & \tcode{<concepts>} \\
\ref{concepts.compare}  & Comparison concepts       &                    \\
\ref{concepts.object}   & Object concepts           &                    \\
\ref{concepts.callable} & Callable concepts         &                    \\
\end{libsumtab}

\rSec1[concepts.equality]{Equality preservation}

\pnum
An expression is \defnx{equality-preserving}{expression!equality-preserving} if,
given equal inputs, the expression results in equal outputs. The inputs to an
expression are the set of the expression's operands. The output of an expression
is the expression's result and all operands modified by the expression.

\pnum
Not all input values need be valid for a given expression; e.g., for integers
\tcode{a} and \tcode{b}, the expression \tcode{a / b} is not well-defined when
\tcode{b} is \tcode{0}. This does not preclude the expression \tcode{a / b}
being equality-preserving. The \term{domain} of an expression is the set of
input values for which the expression is required to be well-defined.

\pnum
Expressions required by this document to be equality-preserving are further
required to be stable: two evaluations of such an expression with the same input
objects are required to have equal outputs absent any explicit intervening
modification of those input objects.
\begin{note}
This requirement allows generic code to reason about the current values of
objects based on knowledge of the prior values as observed via
equality-preserving expressions. It effectively forbids spontaneous changes to
an object, changes to an object from another thread of execution, changes to an
object as side effects of non-modifying expressions, and changes to an object as
side effects of modifying a distinct object if those changes could be observable
to a library function via an equality-preserving expression that is required to
be valid for that object.
\end{note}

\pnum
Expressions declared in a \grammarterm{requires-expression} in this document are
required to be equality-preserving, except for those annotated with the comment
``not required to be equality-preserving.'' An expression so annotated
may be equality-preserving, but is not required to be so.

\pnum
An expression that may alter the value of one or more of its inputs in a manner
observable to equality-preserving expressions is said to modify those inputs.
This document uses a notational convention to specify which expressions declared
in a \grammarterm{requires-expression} modify which inputs: except where
otherwise specified, an expression operand that is a non-constant lvalue or
rvalue may be modified. Operands that are constant lvalues or rvalues are
required to not be modified.

\pnum
Where a \grammarterm{requires-expression} declares an expression that is
non-modifying for some constant lvalue operand, additional variations of that
expression that accept a non-constant lvalue or (possibly constant) rvalue for
the given operand are also required except where such an expression variation is
explicitly required with differing semantics. These
\term{implicit expression variations} are required to meet the semantic
requirements of the declared expression. The extent to which an implementation
validates the syntax of the variations is unspecified.

\pnum
\begin{example}
\begin{codeblock}
template<class T> concept C = requires(T a, T b, const T c, const T d) {
  c == d;           // \#1
  a = std::move(b); // \#2
  a = c;            // \#3
};
\end{codeblock}

For the above example:
\begin{itemize}
\item
Expression \#1 does not modify either of its operands, \#2 modifies both of its
operands, and \#3 modifies only its first operand \tcode{a}.

\item
Expression \#1 implicitly requires additional expression variations that meet
the requirements for \tcode{c == d} (including non-modification), as if the
expressions
\begin{codeblock}
                                            c  ==           b;
          c  == std::move(d);               c  == std::move(b);
std::move(c) ==           d;      std::move(c) ==           b;
std::move(c) == std::move(d);     std::move(c) == std::move(b);

          a  ==           d;                a  ==           b;
          a  == std::move(d);               a  == std::move(b);
std::move(a) ==           d;      std::move(a) ==           b;
std::move(a) == std::move(d);     std::move(a) == std::move(b);
\end{codeblock}
had been declared as well.

\item
Expression \#3 implicitly requires additional expression variations that meet
the requirements for \tcode{a = c} (including non-modification of the second
operand), as if the expressions \tcode{a = b} and \tcode{a = std::move(c)} had
been declared. Expression \#3 does not implicitly require an expression
variation with a non-constant rvalue second operand, since expression \#2
already specifies exactly such an expression explicitly.
\end{itemize}
\end{example}

\pnum
\begin{example}
The following type \tcode{T} meets the explicitly stated syntactic requirements
of concept \tcode{C} above but does not meet the additional implicit
requirements:

\begin{codeblock}
struct T {
  bool operator==(const T&) const { return true; }
  bool operator==(T&) = delete;
};
\end{codeblock}

\tcode{T} fails to meet the implicit requirements of \tcode{C},
so \tcode{T} satisfies but does not model \tcode{C}.
Since implementations are not required to validate the syntax
of implicit requirements, it is unspecified whether an implementation diagnoses
as ill-formed a program that requires \tcode{C<T>}.
\end{example}

\rSec1[concepts.syn]{Header \tcode{<concepts>} synopsis}

\indexheader{concepts}%
\begin{codeblock}
namespace std {
  // \ref{concepts.lang}, language-related concepts
  // \ref{concept.same}, concept \libconcept{same_as}
  template<class T, class U>
    concept same_as = @\seebelow@;

  // \ref{concept.derived}, concept \libconcept{derived_from}
  template<class Derived, class Base>
    concept derived_from = @\seebelow@;

  // \ref{concept.convertible}, concept \libconcept{convertible_to}
  template<class From, class To>
    concept convertible_to = @\seebelow@;

  // \ref{concept.commonref}, concept \libconcept{common_reference_with}
  template<class T, class U>
    concept common_reference_with = @\seebelow@;

  // \ref{concept.common}, concept \libconcept{common_with}
  template<class T, class U>
    concept common_with = @\seebelow@;

  // \ref{concepts.arithmetic}, arithmetic concepts
  template<class T>
    concept integral = @\seebelow@;
  template<class T>
    concept signed_integral = @\seebelow@;
  template<class T>
    concept unsigned_integral = @\seebelow@;
  template<class T>
    concept floating_point = @\seebelow@;

  // \ref{concept.assignable}, concept \libconcept{assignable_from}
  template<class LHS, class RHS>
    concept assignable_from = @\seebelow@;

  // \ref{concept.swappable}, concept \libconcept{swappable}
  namespace ranges {
    inline namespace @\unspec@ {
      inline constexpr @\unspec@ swap = @\unspec@;
    }
  }
  template<class T>
    concept swappable = @\seebelow@;
  template<class T, class U>
    concept swappable_with = @\seebelow@;

  // \ref{concept.destructible}, concept \libconcept{destructible}
  template<class T>
    concept destructible = @\seebelow@;

  // \ref{concept.constructible}, concept \libconcept{constructible_from}
  template<class T, class... Args>
    concept constructible_from = @\seebelow@;

  // \ref{concept.defaultconstructible}, concept \libconcept{default_constructible}
  template<class T>
    concept default_constructible = @\seebelow@;

  // \ref{concept.moveconstructible}, concept \libconcept{move_constructible}
  template<class T>
    concept move_constructible = @\seebelow@;

  // \ref{concept.copyconstructible}, concept \libconcept{copy_constructible}
  template<class T>
    concept copy_constructible = @\seebelow@;

  // \ref{concepts.compare}, comparison concepts
  // \ref{concept.boolean}, concept \libconcept{boolean}
  template<class B>
    concept boolean = @\seebelow@;

  // \ref{concept.equalitycomparable}, concept \libconcept{equality_comparable}
  template<class T>
    concept equality_comparable = @\seebelow@;
  template<class T, class U>
    concept equality_comparable_with = @\seebelow@;

  // \ref{concept.totallyordered}, concept \libconcept{totally_ordered}
  template<class T>
    concept totally_ordered = @\seebelow@;
  template<class T, class U>
    concept totally_ordered_with = @\seebelow@;

  // \ref{concepts.object}, object concepts
  template<class T>
    concept movable = @\seebelow@;
  template<class T>
    concept copyable = @\seebelow@;
  template<class T>
    concept semiregular = @\seebelow@;
  template<class T>
    concept regular = @\seebelow@;

  // \ref{concepts.callable}, callable concepts
  // \ref{concept.invocable}, concept \libconcept{invocable}
  template<class F, class... Args>
    concept invocable = @\seebelow@;

  // \ref{concept.regularinvocable}, concept \libconcept{regular_invocable}
  template<class F, class... Args>
    concept regular_invocable = @\seebelow@;

  // \ref{concept.predicate}, concept \libconcept{predicate}
  template<class F, class... Args>
    concept predicate = @\seebelow@;

  // \ref{concept.relation}, concept \libconcept{relation}
  template<class R, class T, class U>
    concept relation = @\seebelow@;

  // \ref{concept.equiv}, concept \libconcept{equivalence_relation}
  template<class R, class T, class U>
    concept equivalence_relation = @\seebelow@;

  // \ref{concept.strictweakorder}, concept \libconcept{strict_weak_order}
  template<class R, class T, class U>
    concept strict_weak_order = @\seebelow@;
}
\end{codeblock}

\rSec1[concepts.lang]{Language-related concepts}

\rSec2[concepts.lang.general]{General}

\pnum
Subclause \ref{concepts.lang} contains the definition of concepts corresponding to language
features. These concepts express relationships between types, type
classifications, and fundamental type properties.

\rSec2[concept.same]{Concept \cname{same_as}}

\begin{itemdecl}
template<class T, class U>
  concept @\defexposconcept{same-as-impl}@ = is_same_v<T, U>;       // \expos

template<class T, class U>
  concept @\deflibconcept{same_as}@ = @\exposconcept{same-as-impl}@<T, U> && @\exposconcept{same-as-impl}@<U, T>;
\end{itemdecl}

\begin{itemdescr}
\pnum
\begin{note}
\tcode{\libconcept{same_as}<T, U>} subsumes \tcode{\libconcept{same_as}<U, T>} and
vice versa.
\end{note}
\end{itemdescr}

\rSec2[concept.derived]{Concept \cname{derived_from}}

\begin{itemdecl}
template<class Derived, class Base>
  concept @\deflibconcept{derived_from}@ =
    is_base_of_v<Base, Derived> &&
    is_convertible_v<const volatile Derived*, const volatile Base*>;
\end{itemdecl}

\begin{itemdescr}
\pnum
\begin{note}
\tcode{\libconcept{derived_from}<Derived, Base>} is satisfied if and only if
\tcode{Derived} is publicly and unambiguously derived from \tcode{Base}, or
\tcode{Derived} and \tcode{Base} are the same class type ignoring cv-qualifiers.
\end{note}
\end{itemdescr}

\rSec2[concept.convertible]{Concept \cname{convertible_to}}

\pnum
The \libconcept{convertible_to} concept requires an expression of a particular
type and value category to be both implicitly and explicitly convertible to some
other type. The implicit and explicit conversions are required to produce equal
results.

\begin{itemdecl}
template<class From, class To>
  concept @\deflibconcept{convertible_to}@ =
    is_convertible_v<From, To> &&
    requires(From (&f)()) {
      static_cast<To>(f());
    };
\end{itemdecl}

\begin{itemdescr}
\pnum
Let \tcode{test} be the invented function:
\begin{codeblock}
To test(From (&f)()) {
  return f();
}
\end{codeblock}
and let \tcode{f} be a function with no arguments and return type \tcode{From}
such that \tcode{f()} is equality-preserving.
Types \tcode{From} and \tcode{To} model \tcode{\libconcept{convertible_to}<From, To>}
only if:

\begin{itemize}
\item
\tcode{To} is not an object or reference-to-object type, or
\tcode{static_cast<To>(f())} is equal to \tcode{test(f)}.

\item
\tcode{From} is not a reference-to-object type, or

\begin{itemize}
\item
If \tcode{From} is an rvalue reference to a non const-qualified type, the
resulting state of the object referenced by \tcode{f()} after either above
expression is valid but unspecified\iref{lib.types.movedfrom}.

\item
Otherwise, the object referred to by \tcode{f()} is not modified by either above
expression.
\end{itemize}
\end{itemize}
\end{itemdescr}


\rSec2[concept.commonref]{Concept \cname{common_reference_with}}

\pnum
For two types \tcode{T} and \tcode{U}, if \tcode{common_reference_t<T, U>}
is well-formed and denotes a type \tcode{C} such that both
\tcode{\libconcept{convertible_to}<T, C>}
and
\tcode{\libconcept{convertible_to}<U, C>}
are modeled, then \tcode{T} and \tcode{U} share a
\term{common reference type}, \tcode{C}.
\begin{note}
\tcode{C} could be the same as \tcode{T}, or \tcode{U}, or it could be a
different type. \tcode{C} may be a reference type.
\end{note}

\begin{itemdecl}
template<class T, class U>
  concept @\deflibconcept{common_reference_with}@ =
    same_as<common_reference_t<T, U>, common_reference_t<U, T>> &&
    convertible_to<T, common_reference_t<T, U>> &&
    convertible_to<U, common_reference_t<T, U>>;
\end{itemdecl}

\begin{itemdescr}
\pnum
Let \tcode{C} be \tcode{common_reference_t<T, U>}.
Let \tcode{t1} and \tcode{t2} be equality-preserving
expressions\iref{concepts.equality} such that
\tcode{decltype((t1))} and \tcode{decltype((t2))} are each \tcode{T}, and
let \tcode{u1} and \tcode{u2} be equality-preserving expressions such that
\tcode{decltype((u1))} and \tcode{decltype((u2))} are each \tcode{U}.
\tcode{T} and \tcode{U} model \tcode{\libconcept{common_reference_with}<T, U>}
only if:
\begin{itemize}
\item \tcode{C(t1)} equals \tcode{C(t2)} if and only if
  \tcode{t1} equals \tcode{t2}, and
\item \tcode{C(u1)} equals \tcode{C(u2)} if and only if
  \tcode{u1} equals \tcode{u2}.
\end{itemize}

\pnum
\begin{note}
Users can customize the behavior of \libconcept{common_reference_with} by specializing
the \tcode{basic_common_reference} class template\iref{meta.trans.other}.
\end{note}
\end{itemdescr}

\rSec2[concept.common]{Concept \cname{common_with}}

\pnum
If \tcode{T} and \tcode{U} can both be explicitly converted to some third type,
\tcode{C}, then \tcode{T} and \tcode{U} share a \term{common type},
\tcode{C}.
\begin{note}
\tcode{C} could be the same as \tcode{T}, or \tcode{U}, or it could be a
different type. \tcode{C} might not be unique.
\end{note}

\begin{itemdecl}
template<class T, class U>
  concept @\deflibconcept{common_with}@ =
    same_as<common_type_t<T, U>, common_type_t<U, T>> &&
    requires {
      static_cast<common_type_t<T, U>>(declval<T>());
      static_cast<common_type_t<T, U>>(declval<U>());
    } &&
    common_reference_with<
      add_lvalue_reference_t<const T>,
      add_lvalue_reference_t<const U>> &&
    common_reference_with<
      add_lvalue_reference_t<common_type_t<T, U>>,
      common_reference_t<
        add_lvalue_reference_t<const T>,
        add_lvalue_reference_t<const U>>>;
\end{itemdecl}

\begin{itemdescr}
\pnum
Let \tcode{C} be \tcode{common_type_t<T, U>}.
Let \tcode{t1} and \tcode{t2} be
equality-preserving expressions\iref{concepts.equality} such that
\tcode{decltype((t1))} and \tcode{decltype((t2))} are each \tcode{T}, and
let \tcode{u1} and \tcode{u2} be equality-preserving expressions such that
\tcode{decltype((u1))} and \tcode{decltype((u2))} are each \tcode{U}.
\tcode{T} and \tcode{U} model \tcode{\libconcept{common_with}<T, U>}
only if:
\begin{itemize}
\item \tcode{C(t1)} equals \tcode{C(t2)} if and only if
  \tcode{t1} equals \tcode{t2}, and
\item \tcode{C(u1)} equals \tcode{C(u2)} if and only if
  \tcode{u1} equals \tcode{u2}.
\end{itemize}

\pnum
\begin{note}
Users can customize the behavior of \libconcept{common_with} by specializing the
\tcode{common_type} class template\iref{meta.trans.other}.
\end{note}

\end{itemdescr}

\rSec2[concepts.arithmetic]{Arithmetic concepts}

\begin{itemdecl}
template<class T>
  concept @\deflibconcept{integral}@ = is_integral_v<T>;
template<class T>
  concept @\deflibconcept{signed_integral}@ = integral<T> && is_signed_v<T>;
template<class T>
  concept @\deflibconcept{unsigned_integral}@ = integral<T> && !signed_integral<T>;
template<class T>
  concept @\deflibconcept{floating_point}@ = is_floating_point_v<T>;
\end{itemdecl}

\begin{itemdescr}
\pnum
\begin{note}
\libconcept{signed_integral} can be modeled even by types that are
not signed integer types\iref{basic.fundamental}; for example, \tcode{char}.
\end{note}

\pnum
\begin{note}
\libconcept{unsigned_integral} can be modeled even by types that are
not unsigned integer types\iref{basic.fundamental}; for example, \tcode{bool}.
\end{note}
\end{itemdescr}

\rSec2[concept.assignable]{Concept \cname{assignable_from}}

\begin{itemdecl}
template<class LHS, class RHS>
  concept @\deflibconcept{assignable_from}@ =
    is_lvalue_reference_v<LHS> &&
    common_reference_with<const remove_reference_t<LHS>&, const remove_reference_t<RHS>&> &&
    requires(LHS lhs, RHS&& rhs) {
      { lhs = std::forward<RHS>(rhs) } -> same_as<LHS>;
    };
\end{itemdecl}

\begin{itemdescr}
\pnum
Let:
\begin{itemize}
\item \tcode{lhs} be an lvalue that refers to an object \tcode{lcopy} such that
  \tcode{decltype((lhs))} is \tcode{LHS},
\item \tcode{rhs} be an expression such that \tcode{decltype((rhs))} is
  \tcode{RHS}, and
\item \tcode{rcopy} be a distinct object that is equal to \tcode{rhs}.
\end{itemize}
\tcode{LHS} and \tcode{RHS} model
\tcode{\libconcept{assignable_from}<LHS, RHS>} only if

\begin{itemize}
\item \tcode{addressof(lhs = rhs) == addressof(lcopy)}.

\item After evaluating \tcode{lhs = rhs}:

\begin{itemize}
\item \tcode{lhs} is equal to \tcode{rcopy}, unless \tcode{rhs} is a non-const
xvalue that refers to \tcode{lcopy}.

\item If \tcode{rhs} is a non-\tcode{const} xvalue, the resulting state of the
object to which it refers is valid but unspecified\iref{lib.types.movedfrom}.

\item Otherwise, if \tcode{rhs} is a glvalue, the object to which it refers is
  not modified.
\end{itemize}
\end{itemize}

\pnum
\begin{note}
Assignment need not be a total function\iref{structure.requirements};
in particular, if assignment to an object \tcode{x} can result in a modification
of some other object \tcode{y}, then \tcode{x = y} is likely not in the domain
of \tcode{=}.
\end{note}
\end{itemdescr}

\rSec2[concept.swappable]{Concept \cname{swappable}}

\pnum
Let \tcode{t1} and \tcode{t2} be equality-preserving expressions that denote
distinct equal objects of type \tcode{T}, and let \tcode{u1} and \tcode{u2}
similarly denote distinct equal objects of type \tcode{U}.
\begin{note}
\tcode{t1} and \tcode{u1} can denote distinct objects, or the same object.
\end{note}
An operation
\term{exchanges the values} denoted by \tcode{t1} and \tcode{u1} if and only
if the operation modifies neither \tcode{t2} nor \tcode{u2} and:
\begin{itemize}
\item If \tcode{T} and \tcode{U} are the same type, the result of the operation
  is that \tcode{t1} equals \tcode{u2} and \tcode{u1} equals \tcode{t2}.

\item If \tcode{T} and \tcode{U} are different types that model
  \tcode{\libconcept{common_reference_with}<const T\&, const U\&>},
  the result of the operation is that
  \tcode{C(t1)} equals \tcode{C(u2)}
  and
  \tcode{C(u1)} equals \tcode{C(t2)}
  where \tcode{C} is \tcode{common_reference_t<const T\&, const U\&>}.
\end{itemize}

\pnum
\indexlibraryglobal{ranges::swap}%
The name \tcode{ranges::swap} denotes a customization point
object\iref{customization.point.object}. The expression
\tcode{ranges::swap(E1, E2)} for subexpressions \tcode{E1}
and \tcode{E2} is expression-equivalent to an expression
\tcode{S} determined as follows:

\begin{itemize}
\item
  \tcode{S} is \tcode{(void)swap(E1, E2)}\footnote{The name \tcode{swap} is used
  here unqualified.} if \tcode{E1} or \tcode{E2}
  has class or enumeration type\iref{basic.compound} and that expression is valid, with
  overload resolution performed in a context that includes the declaration
\begin{codeblock}
template<class T>
  void swap(T&, T&) = delete;
\end{codeblock}
  and does not include a declaration of \tcode{ranges::swap}.
  If the function selected by overload resolution does not
  exchange the values denoted by
  \tcode{E1} and \tcode{E2},
  the program is ill-formed with no diagnostic required.

\item
  Otherwise, if \tcode{E1} and \tcode{E2}
  are lvalues of array types\iref{basic.compound}
  with equal extent and \tcode{ranges::swap(*E1, *E2)}
  is a valid expression,
  \tcode{S} is \tcode{(void)ranges::swap_ranges(E1, E2)},
  except that
  \tcode{noexcept(S)} is equal to
  \tcode{noexcept(\brk{}ranges::swap(*E1, *E2))}.

\item
  Otherwise, if \tcode{E1} and \tcode{E2} are lvalues of the
  same type \tcode{T} that models \tcode{\libconcept{move_constructible}<T>} and
  \tcode{\libconcept{assignable_from}<T\&, T>},
  \tcode{S} is an expression that exchanges the denoted values.
  \tcode{S} is a constant expression if
  \begin{itemize}
  \item \tcode{T} is a literal type\iref{basic.types},
  \item both \tcode{E1 = std::move(E2)} and \tcode{E2 = std::move(E1)} are
    constant subexpressions\iref{defns.const.subexpr}, and
  \item the full-expressions of the initializers in the declarations
\begin{codeblock}
T t1(std::move(E1));
T t2(std::move(E2));
\end{codeblock}
are constant subexpressions.
  \end{itemize}
  \tcode{noexcept(S)} is equal to
  \tcode{is_nothrow_move_constructible_v<T> \&\& is_nothrow_move_assignable\-_v<T>}.

\item
  Otherwise, \tcode{ranges::swap(E1, E2)} is ill-formed.
  \begin{note}
  This case can result in substitution failure when \tcode{ranges::swap(E1, E2)}
  appears in the immediate context of a template instantiation.
  \end{note}
\end{itemize}

\pnum
\begin{note}
Whenever \tcode{ranges::swap(E1, E2)} is a valid expression, it
exchanges the values denoted by
\tcode{E1} and \tcode{E2} and has type \tcode{void}.
\end{note}

\begin{itemdecl}
template<class T>
  concept @\deflibconcept{swappable}@ = requires(T& a, T& b) { ranges::swap(a, b); };
\end{itemdecl}

\begin{itemdecl}
template<class T, class U>
  concept @\deflibconcept{swappable_with}@ =
    common_reference_with<const remove_reference_t<T>&, const remove_reference_t<U>&> &&
    requires(T&& t, U&& u) {
      ranges::swap(std::forward<T>(t), std::forward<T>(t));
      ranges::swap(std::forward<U>(u), std::forward<U>(u));
      ranges::swap(std::forward<T>(t), std::forward<U>(u));
      ranges::swap(std::forward<U>(u), std::forward<T>(t));
    };
\end{itemdecl}

\pnum
\begin{note}
The semantics of the \libconcept{swappable} and \libconcept{swappable_with}
concepts are fully defined by the \tcode{ranges::swap} customization point.
\end{note}

\pnum
\begin{example}
User code can ensure that the evaluation of \tcode{swap} calls
is performed in an appropriate context under the various conditions as follows:
\begin{codeblock}
#include <cassert>
#include <concepts>
#include <utility>

namespace ranges = std::ranges;

template<class T, std::swappable_with<T> U>
void value_swap(T&& t, U&& u) {
  ranges::swap(std::forward<T>(t), std::forward<U>(u));
}

template<std::swappable T>
void lv_swap(T& t1, T& t2) {
  ranges::swap(t1, t2);
}

namespace N {
  struct A { int m; };
  struct Proxy { A* a; };
  Proxy proxy(A& a) { return Proxy{ &a }; }

  void swap(A& x, Proxy p) {
    ranges::swap(x.m, p.a->m);
  }
  void swap(Proxy p, A& x) { swap(x, p); }      // satisfy symmetry requirement
}

int main() {
  int i = 1, j = 2;
  lv_swap(i, j);
  assert(i == 2 && j == 1);

  N::A a1 = { 5 }, a2 = { -5 };
  value_swap(a1, proxy(a2));
  assert(a1.m == -5 && a2.m == 5);
}
\end{codeblock}
\end{example}

\rSec2[concept.destructible]{Concept \cname{destructible}}

\pnum
The \libconcept{destructible} concept specifies properties of all types,
instances of which can be destroyed at the end of their lifetime, or reference
types.

\begin{itemdecl}
template<class T>
  concept @\deflibconcept{destructible}@ = is_nothrow_destructible_v<T>;
\end{itemdecl}

\begin{itemdescr}
\pnum
\begin{note}
Unlike the \oldconcept{Destructible} requirements~(\tref{cpp17.destructible}), this
concept forbids destructors that are potentially throwing, even if a particular
invocation of the destructor does not actually throw.
\end{note}
\end{itemdescr}

\rSec2[concept.constructible]{Concept \cname{constructible_from}}

\pnum
The \libconcept{constructible_from} concept constrains the initialization of a
variable of a given type with a particular set of argument types.

\begin{itemdecl}
template<class T, class... Args>
  concept @\deflibconcept{constructible_from}@ = destructible<T> && is_constructible_v<T, Args...>;
\end{itemdecl}

\rSec2[concept.defaultconstructible]{Concept \cname{default_constructible}}

\begin{itemdecl}
template<class T>
  inline constexpr bool @\defexposconcept{is-default-initializable}@ = @\seebelow@;  // \expos

template<class T>
  concept @\deflibconcept{default_constructible}@ = constructible_from<T> &&
                                  requires { T{}; } &&
                                  @\exposconcept{is-default-initializable}@<T>;
\end{itemdecl}

\begin{itemdescr}
\pnum
For a type \tcode{T}, \tcode{\exposconcept{is-default-initializable}<T>} is \tcode{true}
if and only if the variable definition
\begin{codeblock}
T t;
\end{codeblock}
is well-formed for some invented variable \tcode{t};
otherwise it is \tcode{false}.
Access checking is performed as if in a context unrelated to \tcode{T}.
Only the validity of the immediate context of the variable initialization is considered.
\end{itemdescr}

\rSec2[concept.moveconstructible]{Concept \cname{move_constructible}}

\begin{itemdecl}
template<class T>
  concept @\deflibconcept{move_constructible}@ = constructible_from<T, T> && convertible_to<T, T>;
\end{itemdecl}

\begin{itemdescr}
\pnum
If \tcode{T} is an object type, then let \tcode{rv} be an rvalue of type
\tcode{T} and \tcode{u2} a distinct object of type \tcode{T} equal to
\tcode{rv}. \tcode{T} models \libconcept{move_constructible} only if

\begin{itemize}
\item After the definition \tcode{T u = rv;}, \tcode{u} is equal to \tcode{u2}.

\item \tcode{T(rv)} is equal to \tcode{u2}.

\item If \tcode{T} is not \tcode{const}, \tcode{rv}'s resulting state is valid
but unspecified\iref{lib.types.movedfrom}; otherwise, it is unchanged.
\end{itemize}
\end{itemdescr}

\rSec2[concept.copyconstructible]{Concept \cname{copy_constructible}}

\begin{itemdecl}
template<class T>
  concept @\deflibconcept{copy_constructible}@ =
    move_constructible<T> &&
    constructible_from<T, T&> && convertible_to<T&, T> &&
    constructible_from<T, const T&> && convertible_to<const T&, T> &&
    constructible_from<T, const T> && convertible_to<const T, T>;
\end{itemdecl}

\begin{itemdescr}
\pnum
If \tcode{T} is an object type, then let \tcode{v} be an lvalue of type
(possibly \tcode{const}) \tcode{T} or an rvalue of type \tcode{const T}.
\tcode{T} models \libconcept{copy_constructible} only if

\begin{itemize}
\item After the definition \tcode{T u = v;}, \tcode{u} is equal to \tcode{v}.

\item \tcode{T(v)} is equal to \tcode{v}.
\end{itemize}

\end{itemdescr}

\rSec1[concepts.compare]{Comparison concepts}

\rSec2[concepts.compare.general]{General}

\pnum
Subclause \ref{concepts.compare} describes concepts that establish relationships and orderings
on values of possibly differing object types.

\rSec2[concept.boolean]{Concept \cname{boolean}}

\pnum
The \libconcept{boolean} concept specifies the requirements on a type that is
usable as a truth value.

\begin{itemdecl}
template<class B>
  concept @\deflibconcept{boolean}@ =
    movable<remove_cvref_t<B>> &&       // (see \ref{concepts.object})
    requires(const remove_reference_t<B>& b1,
             const remove_reference_t<B>& b2, const bool a) {
      { b1 } -> convertible_to<bool>;
      { !b1 } -> convertible_to<bool>;
      { b1 && b2 } -> same_as<bool>;
      { b1 &&  a } -> same_as<bool>;
      {  a && b2 } -> same_as<bool>;
      { b1 || b2 } -> same_as<bool>;
      { b1 ||  a } -> same_as<bool>;
      {  a || b2 } -> same_as<bool>;
      { b1 == b2 } -> convertible_to<bool>;
      { b1 ==  a } -> convertible_to<bool>;
      {  a == b2 } -> convertible_to<bool>;
      { b1 != b2 } -> convertible_to<bool>;
      { b1 !=  a } -> convertible_to<bool>;
      {  a != b2 } -> convertible_to<bool>;
    };
\end{itemdecl}

\pnum
Given a type \tcode{B}, let \tcode{b1} and \tcode{b2} be
lvalues of type \tcode{const remove_reference_t<B>}.
\tcode{B} models \libconcept{boolean} only if

\begin{itemize}
\item \tcode{bool(b1) == !bool(!b1)}.
\item \tcode{(b1 \&\& b2)}, \tcode{(b1 \&\& bool(b2))}, and
      \tcode{(bool(b1) \&\& b2)} are all equal to
      \tcode{(bool(b1) \&\& bool(b2))}, and have the same short-circuit
      evaluation.
\item \tcode{(b1 || b2)}, \tcode{(b1 || bool(b2))}, and
      \tcode{(bool(b1) || b2)} are all equal to
      \tcode{(bool(b1) || bool(b2))}, and have the same short-circuit
      evaluation.
\item \tcode{bool(b1 == b2)}, \tcode{bool(b1 == bool(b2))}, and
      \tcode{bool(bool(b1) == b2)} are all equal to
      \tcode{(bool(b1) == bool(b2))}.
\item \tcode{bool(b1 != b2)}, \tcode{bool(b1 != bool(b2))}, and
      \tcode{bool(bool(b1) != b2)} are all equal to
      \tcode{(bool(b1) != bool(b2))}.
\end{itemize}

\pnum
\begin{example}
The types \tcode{bool}, \tcode{true_type}\iref{meta.type.synop}, and
\tcode{bitset<$N$>::reference}\iref{template.bitset} are \libconcept{boolean}
types. Pointers, smart pointers, and types with only explicit conversions to
\tcode{bool} are not \libconcept{boolean} types.
\end{example}

\rSec2[concept.equalitycomparable]{Concept \cname{equality_comparable}}

\begin{itemdecl}
template<class T, class U>
  concept @\defexposconcept{weakly-equality-comparable-with}@ = // \expos
    requires(const remove_reference_t<T>& t,
             const remove_reference_t<U>& u) {
      { t == u } -> boolean;
      { t != u } -> boolean;
      { u == t } -> boolean;
      { u != t } -> boolean;
    };
\end{itemdecl}

\begin{itemdescr}
\pnum
Given types \tcode{T} and \tcode{U},
let \tcode{t} and \tcode{u} be lvalues of types
\tcode{const remove_reference_t<T>} and
\tcode{const remove_reference_t<U>} respectively.
\tcode{T} and \tcode{U} model
\tcode{\placeholder{weakly-equality-comparable-with}<T, U>} only if
\begin{itemize}
\item \tcode{t == u}, \tcode{u == t}, \tcode{t != u}, and \tcode{u != t}
      have the same domain.
\item \tcode{bool(u == t) ==  bool(t == u)}.
\item \tcode{bool(t != u) == !bool(t == u)}.
\item \tcode{bool(u != t) ==  bool(t != u)}.
\end{itemize}
\end{itemdescr}

\begin{itemdecl}
template<class T>
  concept @\deflibconcept{equality_comparable}@ = @\exposconcept{weakly-equality-comparable-with}@<T, T>;
\end{itemdecl}

\begin{itemdescr}
\pnum
Let \tcode{a} and \tcode{b} be objects of type \tcode{T}.
\tcode{T} models \libconcept{equality_comparable} only if
\tcode{bool(a == b)} is \tcode{true} when \tcode{a} is equal to
\tcode{b}\iref{concepts.equality}, and \tcode{false} otherwise.

\pnum
\begin{note}
The requirement that the expression \tcode{a == b} is equality-preserving
implies that \tcode{==} is transitive and symmetric.
\end{note}
\end{itemdescr}

\begin{itemdecl}
template<class T, class U>
  concept @\deflibconcept{equality_comparable_with}@ =
    equality_comparable<T> && equality_comparable<U> &&
    common_reference_with<const remove_reference_t<T>&, const remove_reference_t<U>&> &&
    equality_comparable<
      common_reference_t<
        const remove_reference_t<T>&,
        const remove_reference_t<U>&>> &&
    @\exposconcept{weakly-equality-comparable-with}@<T, U>;
\end{itemdecl}

\begin{itemdescr}
\pnum
Given types \tcode{T} and \tcode{U},
let \tcode{t} be an lvalue of type \tcode{const remove_reference_t<T>},
\tcode{u} be an lvalue of type \tcode{const remove_reference_t<U>},
and \tcode{C} be:
\begin{codeblock}
common_reference_t<const remove_reference_t<T>&, const remove_reference_t<U>&>
\end{codeblock}
\tcode{T} and \tcode{U} model
\tcode{\libconcept{equality_comparable_with}<T, U>} only if
\tcode{bool(t == u) == bool(C(t) == C(u))}.
\end{itemdescr}

\rSec2[concept.totallyordered]{Concept \cname{totally_ordered}}

\begin{itemdecl}
template<class T>
  concept @\deflibconcept{totally_ordered}@ =
    equality_comparable<T> &&
    requires(const remove_reference_t<T>& a,
             const remove_reference_t<T>& b) {
      { a <  b } -> boolean;
      { a >  b } -> boolean;
      { a <= b } -> boolean;
      { a >= b } -> boolean;
    };
\end{itemdecl}

\begin{itemdescr}
\pnum
Given a type \tcode{T}, let \tcode{a}, \tcode{b}, and \tcode{c} be
lvalues of type \tcode{const remove_reference_t<T>}.
\tcode{T} models \libconcept{totally_ordered} only if

\begin{itemize}
\item Exactly one of \tcode{bool(a < b)}, \tcode{bool(a > b)}, or
      \tcode{bool(a == b)} is \tcode{true}.
\item If \tcode{bool(a < b)} and \tcode{bool(b < c)}, then
      \tcode{bool(a < c)}.
\item \tcode{bool(a >  b) ==  bool(b < a)}.
\item \tcode{bool(a <= b) == !bool(b < a)}.
\item \tcode{bool(a >= b) == !bool(a < b)}.
\end{itemize}

\end{itemdescr}

\begin{itemdecl}
template<class T, class U>
  concept @\deflibconcept{totally_ordered_with}@ =
    totally_ordered<T> && totally_ordered<U> &&
    common_reference_with<const remove_reference_t<T>&, const remove_reference_t<U>&> &&
    totally_ordered<
      common_reference_t<
        const remove_reference_t<T>&,
        const remove_reference_t<U>&>> &&
    equality_comparable_with<T, U> &&
    requires(const remove_reference_t<T>& t,
             const remove_reference_t<U>& u) {
      { t <  u } -> boolean;
      { t >  u } -> boolean;
      { t <= u } -> boolean;
      { t >= u } -> boolean;
      { u <  t } -> boolean;
      { u >  t } -> boolean;
      { u <= t } -> boolean;
      { u >= t } -> boolean;
    };
\end{itemdecl}

\begin{itemdescr}
\pnum
Given types \tcode{T} and \tcode{U},
let \tcode{t} be an lvalue of type \tcode{const remove_reference_t<T>},
\tcode{u} be an lvalue of type \tcode{const remove_reference_t<U>},
and \tcode{C} be:
\begin{codeblock}
common_reference_t<const remove_reference_t<T>&, const remove_reference_t<U>&>
\end{codeblock}
\tcode{T} and \tcode{U} model
\tcode{\libconcept{totally_ordered_with}<T, U>} only if

\begin{itemize}
\item \tcode{bool(t <  u) == bool(C(t) <  C(u)).}
\item \tcode{bool(t >  u) == bool(C(t) >  C(u)).}
\item \tcode{bool(t <= u) == bool(C(t) <= C(u)).}
\item \tcode{bool(t >= u) == bool(C(t) >= C(u)).}
\item \tcode{bool(u <  t) == bool(C(u) <  C(t)).}
\item \tcode{bool(u >  t) == bool(C(u) >  C(t)).}
\item \tcode{bool(u <= t) == bool(C(u) <= C(t)).}
\item \tcode{bool(u >= t) == bool(C(u) >= C(t)).}
\end{itemize}
\end{itemdescr}

\rSec1[concepts.object]{Object concepts}

\pnum
This subclause describes concepts that specify the basis of the
value-oriented programming style on which the library is based.

\begin{itemdecl}
template<class T>
  concept @\deflibconcept{movable}@ = is_object_v<T> && move_constructible<T> &&
                    assignable_from<T&, T> && swappable<T>;
template<class T>
  concept @\deflibconcept{copyable}@ = copy_constructible<T> && movable<T> && assignable_from<T&, const T&>;
template<class T>
  concept @\deflibconcept{semiregular}@ = copyable<T> && default_constructible<T>;
template<class T>
  concept @\deflibconcept{regular}@ = semiregular<T> && equality_comparable<T>;
\end{itemdecl}

\begin{itemdescr}
\pnum
\begin{note}
The \libconcept{semiregular} concept is modeled by types that behave similarly
to built-in types like \tcode{int}, except that they might not
be comparable with \tcode{==}.
\end{note}

\pnum
\begin{note}
The \libconcept{regular} concept is modeled by types that behave similarly to
built-in types like \tcode{int} and that are comparable with
\tcode{==}.
\end{note}
\end{itemdescr}

\rSec1[concepts.callable]{Callable concepts}

\rSec2[concepts.callable.general]{General}

\pnum
The concepts in subclause \ref{concepts.callable} describe the requirements on function
objects\iref{function.objects} and their arguments.

\rSec2[concept.invocable]{Concept \cname{invocable}}

\pnum
The \libconcept{invocable} concept specifies a relationship between a callable
type\iref{func.def} \tcode{F} and a set of argument types \tcode{Args...} which
can be evaluated by the library function \tcode{invoke}\iref{func.invoke}.

\begin{itemdecl}
template<class F, class... Args>
  concept @\deflibconcept{invocable}@ = requires(F&& f, Args&&... args) {
    invoke(std::forward<F>(f), std::forward<Args>(args)...); // not required to be equality-preserving
  };
\end{itemdecl}

\begin{itemdescr}
\pnum
\begin{example}
A function that generates random numbers can model \libconcept{invocable},
since the \tcode{invoke} function call expression is not required to be
equality-preserving\iref{concepts.equality}.
\end{example}
\end{itemdescr}

\rSec2[concept.regularinvocable]{Concept \cname{regular_invocable}}

\begin{itemdecl}
template<class F, class... Args>
  concept @\deflibconcept{regular_invocable}@ = invocable<F, Args...>;
\end{itemdecl}

\begin{itemdescr}
\pnum
The \tcode{invoke} function call expression shall be
equality-preserving\iref{concepts.equality} and
shall not modify the function object or the arguments.
\begin{note}
This requirement supersedes the annotation in the definition of
\libconcept{invocable}.
\end{note}

\pnum
\begin{example}
A random number generator does not model \libconcept{regular_invocable}.
\end{example}

\pnum
\begin{note}
The distinction between \libconcept{invocable} and \libconcept{regular_invocable}
is purely semantic.
\end{note}
\end{itemdescr}

\rSec2[concept.predicate]{Concept \cname{predicate}}

\begin{itemdecl}
template<class F, class... Args>
  concept @\deflibconcept{predicate}@ = regular_invocable<F, Args...> && boolean<invoke_result_t<F, Args...>>;
\end{itemdecl}

\rSec2[concept.relation]{Concept \cname{relation}}

\begin{itemdecl}
template<class R, class T, class U>
  concept @\deflibconcept{relation}@ =
    predicate<R, T, T> && predicate<R, U, U> &&
    predicate<R, T, U> && predicate<R, U, T>;
\end{itemdecl}

\rSec2[concept.equiv]{Concept \cname{equivalence_relation}}

\begin{itemdecl}
template<class R, class T, class U>
  concept @\deflibconcept{equivalence_relation}@ = relation<R, T, U>;
\end{itemdecl}

\begin{itemdescr}
\pnum
A \libconcept{relation} models \libconcept{equivalence_relation} only if
it imposes an equivalence relation on its arguments.
\end{itemdescr}

\rSec2[concept.strictweakorder]{Concept \cname{strict_weak_order}}

\begin{itemdecl}
template<class R, class T, class U>
  concept @\deflibconcept{strict_weak_order}@ = relation<R, T, U>;
\end{itemdecl}

\begin{itemdescr}
\pnum
A \libconcept{relation} models \libconcept{strict_weak_order} only if
it imposes a \term{strict weak ordering} on its arguments.

\pnum
The term
\term{strict}
refers to the
requirement of an irreflexive relation (\tcode{!comp(x, x)} for all \tcode{x}),
and the term
\term{weak}
to requirements that are not as strong as
those for a total ordering,
but stronger than those for a partial
ordering.
If we define
\tcode{equiv(a, b)}
as
\tcode{!comp(a, b) \&\& !comp(b, a)},
then the requirements are that
\tcode{comp}
and
\tcode{equiv}
both be transitive relations:

\begin{itemize}
\item
\tcode{comp(a, b) \&\& comp(b, c)}
implies
\tcode{comp(a, c)}
\item
\tcode{equiv(a, b) \&\& equiv(b, c)}
implies
\tcode{equiv(a, c)}
\end{itemize}

\pnum
\begin{note}
Under these conditions, it can be shown that
\begin{itemize}
\item
\tcode{equiv}
is an equivalence relation,
\item
\tcode{comp}
induces a well-defined relation on the equivalence
classes determined by
\tcode{equiv}, and
\item
the induced relation is a strict total ordering.
\end{itemize}
\end{note}
\end{itemdescr}
